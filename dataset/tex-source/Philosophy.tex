\textbf{What is Philosophy?}\footnote{This chapter was originally
  published under the title ``Reason, Expression, and the Philosophic
  Enterprise,'' in \emph{What Is Philosophy?}, C.P. Ragland and Sarah
  Heidt (eds.), Yale University Press, 2001, pp. 74-95.}

1. In this essay, I want to address the question: "What is philosophy?"

We might to begin with acknowledge a distinction between things that
have \emph{natures} and things that have \emph{histories}. Physical
things such as electrons and aromatic compounds would be paradigmatic of
the first class, while cultural formations such as English Romantic
poetry and Ponzi schemes would be paradigmatic of the second. Applied to
the case at hand, this distinction would surely place philosophy on the
side of things that have histories. But now we might ask: Does
philosophy differ in this respect from physics, chemistry, or biology?
Physical, chemical, and biological \emph{things} have natures rather
than histories, but what about the disciplines that define and study
them? Should physics itself be thought of as something that has a
nature, or as something that has a history? Concluding the latter is
giving a certain kind of pride of place to the historical. For it is in
effect treating the \emph{distinction} between things that have natures
and things that have histories, between things studied by the
Naturwissenschaften and things studied by the Geisteswissenschaften, as
itself a cultural formation: the sort of thing that itself has a history
rather than a nature. And from here it is a short step (though not, to
be sure, an obligatory one) to the thought that natures themselves are
the sort of thing that have a history; certainly the \emph{concepts}
\underline{electron} and \underline{aromatic compound} are that sort of
thing. At this point the door is opened to a thorough-going historicism.
It is often thought that this is the point to which Hegel---one of my
particular heroes---brought us. I think that thought is correct, as far
as it goes, but that we go very wrong if we think that that is where
Hegel left us.

To say that philosophy is, at least to begin with, to be understood as
the sort of thing that has a history rather than a nature is to
foreground the way in which what deserves to be counted as distinctively
philosophical activity answers to what has actually been done by those
we recognize as precedential, tradition-transforming philosophers. One
of Hegel's deepest and most important insights, I think, is indeed that
the determinate contentfulness of any universal---in this case, the
concept of philosophy---can only be understood in terms of the process
by which it incorporates the contingencies of the particulars to which
it has actually been applied. But he goes on from there to insist that
it is in each case the responsibility of those of us who are heirs to
such a conceptual tradition to see to it that is a \emph{rational}
tradition---that the distinction it embodies and enforces between
correct and incorrect applications of a concept can be \emph{justified},
that applying it in one case and withholding application in another is
something for which \emph{reasons} can be given. It is only insofar as
we can do that that we are entitled to understand what we are doing as
applying \emph{concepts}. We fulfill that obligation by rationally
reconstructing the tradition, finding a coherent, cumulative trajectory
through it that reveals it as expressively progressive---as the gradual
unfolding into greater explicitness of commitments that can be seen
retrospectively as always already having been implicit in it. That is,
it is our job to rewrite the history so as to discover in it the
revelation of what then retrospectively appears as an antecedent nature.
Hegel balances the insight that even natures have histories by seeing
rationality itself as imposing the obligation to construe histories as
revelatory of natures.

The aim is to pick out a sequence of precedential instances or
applications of a concept that amount to the delineation of a content
for the concept, much as a judge at common law is obliged to do.
\emph{Making} the tradition rational, is not independent of the labor of
concretely \emph{taking} it to be so. It is a criterion of adequacy of
each such Whiggish rewriting of our disciplinary history that it create
and display continuity and progress by its systematic inclusions and
exclusions. The discontinuities that correspond to shifts of topic, the
forgetting of lessons, and the degeneration of research programs are
invisible from within each such telling; but those differences live on
in the spaces between the tellings. Each generation redefines its
subject by offering a new retrospective reading of its characteristic
concerns and hard-won lessons.\footnote{I am describing, of course, for
  the concept \underline{philosophy} an exercise of the sort of
  recollective rationality (Hegel's ``Erinnerung'') considered for
  ordinary determinate empirical concepts in Chapter Three.} But also,
at any one time there will be diverse interpretations, complete with
rival canons, competing designations of heroes, and accounts of their
heroic feats. Making canons and baking traditions out of the rich
ingredients bequeathed us by our discursive predecessors is a game that
all can play.

In this chapter, I am going to sketch one such perspective on what
philosophers do---discern a nature as revealed by the history.

Ours is a broadly cognitive enterprise---I say `\emph{broadly}
cognitive' to indicate that I mean that philosophers aim at a kind of
\emph{understanding}, not, more narrowly, at a kind of \emph{knowledge}.
To specify the distinctive sort of understanding that is the
characteristic goal of philosophers' writing is to say what
distinguishes that enterprise from that of other sorts of constructive
seekers of understanding, such as novelists and scientific theorists. I
want to do so by focusing not on the peculiar genre of nonfiction
creative writing by which philosophical understanding is typically
conveyed (though I think that subject is worthy of consideration), but
rather on what is distinctive about the understanding itself: both its
particular topic, and its characteristic goal. Philosophy is a
self-reflexive enterprise: understanding is not only the \emph{goal} of
philosophical inquiry, but its \emph{topic} as well. \emph{We} are its
topic; but it is us specifically as \emph{understanding} creatures:
\emph{discursive} beings, makers and takers of \emph{reasons}, seekers
and speakers of \emph{truth}. Seeing philosophy as addressing the nature
and conditions of our rationality is, of course, a very traditional
outlook---so traditional, indeed, that it is liable to seem quaint and
oldfashioned. I'll address this issue later, remarking now only that
rationalism is one thing, and intellectualism another: pragmatists, too,
are concerned with the practices of giving and asking for reasons.

I understand the task of philosophers to have as a central element the
explication of concepts---or, put slightly more carefully, the
development and application of expressive tools with which to make
explicit what is implicit in the use of concepts. When I say
"explication of concepts", it is hard not to hear "analysis of
meanings." There are obviously affinities between my specification and
that which defined the concern specifically of "analytic philosophy" in
the middle years of this century. Indeed, I intend, \emph{inter alia},
to be saying what was right about that conception. But what I have in
mind is different in various ways. \emph{Explication}, making explicit,
is not the same as \emph{analysis}, at least as that notion was
classically conceived. As I use the term, for instance, we have no more
privileged access to the contents of our concepts than we do to the
facts we use them to state; the concepts and the facts are two sides of
one coin.

But the most important difference is that where analysis of meanings is
a fundamentally \emph{conservative} enterprise (consider the paradox of
analysis), I see the point of explicating concepts rather to be opening
them up to rational \emph{criticism}. The rational enterprise, the
practice of giving and asking for reasons that lies at the heart of
discursive activity, requires not only criticizing \emph{beliefs}, as
false or unwarranted, but also criticizing \emph{concepts}. Defective
concepts distort our thought and constrain us by limiting the
propositions and plans we can entertain as candidates for endorsement in
belief and intention. This constraint operates behind our backs, out of
our sight, since it limits what we are so much as capable of being aware
of. Philosophy, in developing and applying tools for the rational
criticism of concepts, seeks to free us from these fetters, by bringing
the distorting influences out into the light of conscious day, exposing
the commitments implicit in our concepts as vulnerable to rational
challenge and debate.

2. The first thing to understand about concepts is that
\underline{concept} is a \emph{normative} concept. This is a lesson we
owe ultimately to Kant---the great, gray mother of us all. Kant saw us
above all as traffickers in concepts. In fact, in a strict sense,
\emph{all} that kantian rational creatures can do is to apply concepts.
For that is the genus he took to comprise both \emph{judgment} and
\emph{action}, our theoretical activity and our practical activity. One
of Kant's great innovations was his view that what in the first instance
distinguishes judgments and actions from the mere behavior of denizens
of the realm of nature is that they are things that we are in a
distinctive sense \emph{responsible} for. They express
\emph{commitments} of ours. The norms or rules that determine what we
have committed ourselves to, what we have made ourselves responsible
for, by making a judgment or performing an action, Kant calls
`concepts'. Judging and acting involves undertaking commitments whose
credentials are always potentially at issue. That is, the commitments
embodied in judgments and actions are ones we may or may not be
\emph{entitled} to, so that the question of whether they are
\emph{correct}, whether they are commitments we \emph{ought} to
acknowledge and embrace, can always be raised. One of the forms taken by
the responsibility we undertake in judging and acting is the
responsibility to give reasons that justify the judgment or the action.
And the rules that are the concepts we apply in judging and acting
determine what would count as a reason for the judgment and the action.

Commitment, entitlement, responsibility---these are all normative
notions. Kant replaces the \emph{ontological} distinction between the
physical and the mental with the \emph{deontological} distinction
between the realm of nature and the realm of freedom: the distinction
between things that merely act regularly and things that are subject to
distinctively normative sorts of assessment.

Thus for Kant the great philosophical questions are questions about the
source and nature of normativity---of the bindingness or validity
{[}Gültigkeit{]} of conceptual rules. Descartes had bequeathed to his
successors a concern for \emph{certainty}: a matter of our grip on
concepts and ideas---paradigmatically, whether we have a hold on them
that is clear and distinct. Kant bequeaths to his successors a concern
rather for \emph{necessity}: a matter of the grip concepts have on us,
the way they bind or oblige us. `Necessary' {[}notwendig{]} for Kant
just means ``according to a rule''. (That is why he is willing to speak
of moral and natural necessity as species of a genus.) The important
lesson he takes Hume to have taught isn't about the threat of
skepticism, but about how empirical knowledge is unintelligible if we
insist on merely \emph{de}scribing how things in fact \emph{are},
without moving beyond that to \emph{pre}scribing how they \emph{must}
be, according to causal rules, and how empirical motivation (and so
agency) is unintelligible if we stay at the level of `\emph{is}' and
eschew reference to the `\emph{ought}'s that outrun what merely is.
Looking farther back, Kant finds ``the celebrated Mr. Locke''
sidetracked into a mere ``physiology of the understanding''---the
tracing of causal antecedents of thought in place of its justificatory
antecedents---through a failure to appreciate the essentially normative
character of claims to knowledge. But Kant takes the whole Enlightenment
to be animated by an at least implicit appreciation of this point. For
mankind's coming into its intellectual and spiritual majority and
maturity consists precisely in taking the sort of personal
responsibility for its commitments, both doxastic and practical,
insisted upon already by Descartes' meditator.

This placing of normativity at the center of philosophical concern is
the reason behind another of Kant's signal innovations: the pride of
place he accords to \emph{judgment}. In a sharp break with tradition, he
takes it that the smallest unit of experience, and hence of awareness,
is the judgment. This is because judgments, applications of concepts,
are the smallest unit for which knowers can be \emph{responsible}.
Concepts by themselves don't express commitments; they only determine
what commitments would be undertaken if they were applied. (Frege will
express this kantian point by saying that judgeable contents are the
smallest unit to which pragmatic force---paradigmatically the
assertional force that consists in the assertor undertaking a special
kind of commitment---can attach. Wittgenstein will distinguish sentences
from terms and predicates as the smallest expressions whose
free-standing utterance can be used to make a move in a language game.)
The most general features of Kant's understanding of the form of
judgment also derive from its role as a unit of responsibility. The ``I
think'' that can accompany all representations (hence being, in its
formality, the emptiest of all) is the formal shadow of the
transcendental unity of apperception, the locus of responsibility
determining a coresponsibility class of concept-applications (including
actions), what is responsible \emph{for} its judgments. The objective
correlate of this subjective aspect of the form of judgment is the
``object=X'' to which the judgment is directed, the formal shadow of
what the judgment makes the knower responsible \emph{to}.

I think that philosophy is the study of us as creatures who judge and
act, that is, as discursive, concept-using creatures. And I think that
Kant is right to emphasize that understanding what we do in these terms
is attributing to us various kinds of normative status, taking us to be
subject to distinctive sorts of normative appraisal. So a central
philosophical task is understanding this fundamental normative dimension
within which we dwell. Kant's own approach to this issue, developing
themes from Rousseau, is based on the thought that genuinely normative
authority (constraint by norms) is distinguished from causal power
(constraint by facts) in that it binds only those who \emph{acknowledge}
it as binding. Because one is subject only to that authority one
subjects oneself to, the normative realm can be understood equally as
the realm of \emph{freedom}. So being constrained by norms is not only
compatible with freedom---properly understood, it can be seen to be what
freedom consists in. I don't know of a thought that is deeper, more
difficult, or more important than this.

3. Kant's most basic idea, I said, is that judgment and action are
things we are in a distinctive way \emph{responsible} for. What does it
mean to be responsible for them? I think the kind of responsibility in
question should be understood to be task responsibility: the
responsibility to do something. What (else) do judging and acting oblige
us to do? The commitments we undertake by applying concepts in
particular circumstances---by judging and acting---are ones we may or
may not be entitled to, according to the rules (norms) implicit in those
concepts. Showing that we are entitled by the rules to apply the concept
in a particular case is \emph{justifying} the commitment we undertake
thereby, offering \emph{reasons} for it. That is what we are responsible
for, the practical content of our conceptual commitments. In undertaking
a conceptual commitment one renders oneself in principle liable to
demands for reasons. The normative appraisal to which we subject
ourselves in judging and acting is appraisal of our reasons. Further,
offering a reason for the application of a concept is always applying
another concept: making or rehearsing another judgment or undertaking or
acknowledging another practical commitment (Kant's ``adopting a
maxim''). Conceptual commitments both serve as and stand in need of
reasons. The normative realm inhabited by creatures who can judge and
act is not only the realm of freedom, it is the realm of
reason.\footnote{This story is told in more detail in Chapter One.}

Understanding the norms for correct application that are implicit in
concepts requires understanding the role those concepts play in
reasoning: what (applications of concepts) count as reasons for the
application of that concept, and what (applications of concepts) the
application of that concept counts as a reason for. For apart from such
understanding, one cannot fulfill the responsibility one undertakes by
making a judgment or performing an action. So what distinguishes
concept-using creatures from others is that we know our way around the
\emph{space of reasons}. Grasping or understanding a concept just is
being able practically to place it in a network of inferential
relations: to know what is evidence for or against its being properly
applied to a particular case, and what its proper applicability to a
particular case counts as evidence for or against. Our capacity to know
(or believe) \emph{that} something is the case depends on our having a
certain kind of know \emph{how}: the ability to tell what is a reason
for what.

The cost of losing sight of this point is to assimilate genuinely
conceptual activity, judging and acting, too closely to the behavior of
mere animals---creatures who do not live and move and have their being
in the normative realm of freedom and reason. We share with other
animals (and for that matter, with bits of automatic machinery) the
capacity reliably to respond differentially to various kinds of stimuli.
We, like they, can be understood as classifying stimuli as being of
certain kinds, insofar as we are disposed to produce different
repeatable sorts of responses to those stimuli. We can respond
differentially to red things by uttering the noise ``That is red.'' A
parrot could be trained to do this, as pigeons are trained to peck at a
different button when shown a red figure than when shown a green one.
The empiricist tradition is right to emphasize that our capacity to have
empirical knowledge begins with and crucially depends on such reliable
differential responsive dispositions. But though the story begins with
this sort of classification, it does not end there. For the rationalist
tradition is right to emphasize that our classificatory responses count
as applications of concepts, and hence as so much as candidates for
knowledge, only in virtue of their role in reasoning. The crucial
difference between the parrot's utterance of the noise ``That is red,''
and the (let us suppose physically indistinguishable) utterance of a
human reporter is that for the latter, but not the former, the utterance
has the practical significance of making a claim. Doing that is taking
up a normative stance of a kind that can serve as a premise from which
to draw conclusions. That is, it can serve as a reason for taking up
other stances. And further, it is a stance that itself can stand in need
of reasons, at least if challenged by the adoption of other,
incompatible stances. Where the parrot is merely responsively sounding
off, the human counts as applying a concept just insofar as she is
understood as making a move in a game of giving and asking for reasons.

The most basic point of Sellars' rationalist critique of empiricism in
his masterwork ``Empiricism and the Philosophy of Mind,'' is that even
the \emph{non}inferentially elicited perceptual judgments that the
empiricist rightly appreciates as forming the empirical basis for our
knowledge can count as judgments (applications of concepts) at all only
insofar as they are \emph{inferentially} articulated. Thus the idea that
there could be an autonomous language game (a game one could play though
one played no other) consisting entirely of noninferentially elicited
reports---whether of environing stimuli or of the present contents of
one's own mind---is a radical mistake. To apply any concepts
\emph{non}inferentially, one must be able also to apply concepts
inferentially. For it is an essential feature of concepts that their
applications can both serve as and stand in need of reasons. Making a
report or a perceptual judgment is doing something that essentially, and
not just accidentally, has the significance of making available a
premise for reasoning. Learning to observe requires learning to infer.
Experience and reasoning are two sides of one coin, two capacities
presupposed by concept use that are in principle intelligible only in
terms of their relations to each other.\footnote{Chapter Seven develops
  this theme further.}

To claim that what distinguishes specifically conceptual classification
from classification merely by differential responsive disposition is the
inferential articulation of the response---that applications of concepts
are essentially what can both serve as and stand in need of reasons---is
to assign the game of giving and asking for reasons a preeminent place
among discursive practices. For it is to say that what makes a practice
\emph{discursive} in the first place is that it incorporates
reason-giving practices. Now of course there are many things one can do
with concepts besides using them to argue and to justify. And it has
seemed perverse to some post-Enlightenment thinkers in any way to
privilege the rational, cognitive dimension of language use. But if the
tradition I have been sketching is right, the capacity to use concepts
in all the other ways explored and exploited by the artists and writers
whose imaginative enterprises have rightly been admired by romantic
opponents of logocentrism is parasitic on the prosaic inferential
practices in virtue of which we are entitled to see concepts as in play
in the first place. The game of giving and asking for reasons is not
just one game among others one can play with language. It is the game in
virtue of the playing of which what one has qualifies as \emph{language}
(or thought) at all. I am here disagreeing with Wittgenstein, when he
claims that ``language has no downtown.'' On my view, it does, and that
downtown (the region around which all the rest of discourse is arrayed
as dependent suburbs, is the practices of giving and asking for reasons.
This is a kind of linguistic \emph{rationalism}. `Rationalism' in this
sense does not entail intellectualism, the doctrine that every
\emph{implicit} mastery of a propriety of practice is ultimately to be
explained by appeal to a prior \emph{explicit} grasp of a principle. It
is entirely compatible with the sort of pragmatism that sees things the
other way around.

4. As I am suggesting that we think of them, concepts are broadly
inferential norms that implicitly govern practices of giving and asking
for reasons. Dummett has suggested a useful model for thinking about the
inferential articulation of conceptual contents. Generalizing from the
model of meaning Gentzen introduces for sentential operators, Dummett
suggests that we think of the use of any expression as involving two
components: the circumstances in which it is appropriately used and the
appropriate consequences of such use. Since our concern is with the
application of the concepts expressed by using linguistic expressions,
we can render this as the circumstances of appropriate application of
the concept, and the appropriate consequences of such application---that
is, what follows from the concept's being applicable.

Some of the circumstances and consequences of applicability of a concept
may be inferential in nature. For instance, one of the circumstances of
appropriate application of the concept \underline{red} is that this
concept is applicable wherever the concept \underline{scarlet} is
applicable. And to say that is just another way of saying that the
inference from ``X is scarlet,'' to ``X is red,'' is a good one. And
similarly, one of the consequences of the applicability of the concept
\underline{red} is the applicability of the concept \underline{colored}.
And to say that is just another way of saying that the inference from
``X is red,'' to ``X is colored,'' is a good one. But concepts like
\underline{red} also have \emph{non}inferential circumstances of
applicability, such as the visible presence of red things. And concepts
such as \underline{unjust} have noninferential consequences of
application---that is, they can make it appropriate to \emph{do} (or not
do) something, to make another claim true, not just to \emph{say} or
judge that it is true.

Even the immediately empirical concepts of \emph{observables}, which
have noninferential \emph{circumstances} of application and the
immediately practical \emph{evaluative} concepts, which have
noninferential \emph{consequences} of application, however, can be
understood to have contents that are inferentially articulated. For all
concepts incorporate an implicit commitment to the propriety of the
inference from their circumstances to their consequences of application.
One cannot use the concept \underline{red} as including the
circumstances and consequences mentioned above without committing
oneself to the correctness of the inference from ``X is scarlet,'' to
``X is colored.'' So we might decompose the norms that govern the use of
concepts into three components: circumstances of appropriate
application, appropriate consequences of application, and the propriety
of an inference from the circumstances to the consequences. I would
prefer to understand the inferential commitment expansively, as
including the circumstances and consequences it relates, and so as
comprising all three normative elements.

I suggested at the outset that we think of philosophy as charged with
producing and deploying tools for the criticism of concepts. The key
point here is that concepts may incorporate defective inferences.
Dummett offers this suggestive example:

\begin{quote}
A simple case would be that of a pejorative term, e.g. 'Boche'. The
conditions for applying the term to someone is that he is of German
nationality; the consequences of its application are that he is
barbarous and more prone to cruelty than other Europeans. We should
envisage the connections in both directions as sufficiently tight as to
be involved in the very meaning of the word: neither could be severed
without altering its meaning. Someone who rejects the word does so
because he does not want to permit a transition from the grounds for
applying the term to the consequences of doing so.\footnote{Dummett,
  Frege: Philosophy of Language {[}Harper and Row, New York, 1973{]} p.
  454.}
\end{quote}

(It is useful to focus on a French epithet from the first world war,
because we are sufficiently removed from its practical effect to be able
to get a theoretical grip on how it works. But the thought should go
over \emph{mutatis mutandis} for pejoratives in current circulation.)
Dummett's idea is that if you do not accept as correct the inference
from German nationality to an unusual disposition to barbarity and
cruelty, you can only reject the word. You cannot deny that there are
any Boche, for that is just denying that the circumstances of
application are ever satisfied, that is, that there are any Germans. And
you cannot admit that there are Boche but deny that they are disposed to
barbarity and cruelty (this is the ``Some of my best friends are
Boche,'' ploy), since that is just taking back in one breath what one
has asserted just before. Any use of the term commits the user to the
inference that is curled up, implicitly, in it. (At Oscar Wilde's trial
the prosecutor read out some passages from the Importance of Being
Earnest and said ``I put it to you, Mr. Wilde, that this is blasphemy.
Is it? Yes or no?'' Wilde replied just as he ought on the account I am
urging: ``Sir, `blasphemy' is not one of my words.''\footnote{Of course,
  being right on this point didn't keep Wilde out of trouble, anymore
  than it did Salman Rushdie.})

Although they are perhaps among the most dangerous, it is not just
highly-charged words, words that couple `descriptive' circumstances of
application with `evaluative' consequences of application that
incorporate inferences of which we may need to be critical. The use of
\emph{any} expression involves commitment to the propriety of the
inference from its circumstances to its consequences of application.
These are almost never logically valid inferences. On the contrary, they
are what Sellars called ``material'' inferences: inferences that
articulate the content of the concept expressed. Classical disputes
about the nature of personal identity, for instance, can be understood
as taking the form of arguments about the propriety of such a material
inference. We can agree, we may suppose, about the more or less forensic
consequences of application of the concept ``same person,'' having in
mind its significance for attributions of (co-)responsibility. When we
disagree about the circumstances of application that should be paired
with it---for instance whether bodily or neural continuity, or the
psychological continuity of memory count for more---we are really
disagreeing about the correctness of the inference from the obtaining of
these conditions to the ascription of responsibility. The question about
what is the correct concept is a question about which inferences to
endorse. I think it is helpful to think about a great number of the
questions we ask about other important concepts in these same terms: as
having the form of queries about what inferences from circumstances to
consequences of application we ought to acknowledge as correct, and why.
Think in these terms about such very abstract concepts as
\underline{morally wrong}, \underline{just}, \underline{beautiful},
\underline{true}, \underline{explain}, \underline{know}, or
\underline{prove}, and again about `thicker' ones such as
\underline{unkind}, \underline{cruel}, \underline{elegant},
\underline{justify}, and \underline{understand}.

The use of any of these concepts involves a material inferential
commitment: commitment to the propriety of a substantial inferential
move from the circumstances in which it is appropriate to apply the
concept to the consequences of doing so. The concepts are substantive
just because the inferences they incorporate are. Exactly this
commitment becomes invisible, however, if one conceives conceptual
content in terms of \emph{truth conditions}. For the idea of truth
conditions is the idea of a single set of conditions that are at once
necessary and sufficient for the application of the concept. The idea of
individually necessary conditions that are also jointly sufficient is
the idea of a set of consequences of application that can also serve as
circumstances of application. Thus the circumstances of application are
understood as already including the consequences of application, so that
no endorsement of a substantive inference is involved in using the
concept. The concept of concepts like this is not incoherent. It is the
ideal of \emph{logical} or \emph{formal} concepts. Thus it is a
criterion of adequacy for introducing logical connectives that they be
inferentially conservative: that their introduction and elimination
rules be so related that they permit no new inferences involving only
the old vocabulary. But it is a bad idea to take this model of the
relation between circumstances and consequences of application of
logical vocabulary and extend it to encompass also the substantively
contentful \emph{non}logical concepts that are the currency in which
most of our cognitive and practical transactions are conducted.

It is a bad idea because of its built-in conservatism. Understanding
meaning or conceptual content in terms of truth
conditions---individually necessary and jointly sufficient
conditions---squeezes out of the picture the substantive inferential
commitment implicit in the use of any nonlogical concept. But it is
precisely those inferential commitments that are subject to
\emph{criticism} in the light of substantive collateral beliefs. If one
does not believe that Germans are distinctively barbarous or prone to
cruelty, then one must not use the concept \underline{Boche}, just
\emph{because} one does not endorse the substantive material inference
it incorporates. On the other model, this diagnosis is not available.
The most one can say is that one does not know how to specify truth
conditions for the concept. But just what is objectionable about it and
why does not appear from this theoretical perspective. . Criticism of
concepts is always criticism of the inferential connections. For
criticizing whether all the individually sufficient conditions
(circumstances) ``go together'', i.e. are circumstances of application
of one concept, just is wondering whether they all have the same
consequences of application (and similarly for wondering whether the
consequences of application all ``go together'').

5. When we think of conceptual contents in the way I am recommending, we
can see not only how beliefs can be used to criticize concepts, but also
how concepts can be used to criticize beliefs. For it is the material
inferences incorporated in our concepts that we use to elaborate the
antecedents and consequences of various candidates for belief---to tell
what we would be committing ourselves to, what would entitle us to those
commitments, what would be incompatible with them, and so on. Once it is
accepted that the inferential norms implicit in our concepts are in
principle as revisable in the light of evidence as particular beliefs,
conceptual and empirical authority appear as two sides of one coin.
Rationally justifying our concepts depends on finding out about how
things are---about what actually follows from what---as is most evident
in the case of massively defective concepts such as \underline{Boche}.

Adjusting our beliefs in the light of the connections among them
dictated by our concepts, and our concepts in the light of our evidence
for the substantive beliefs presupposed by the inferences they
incorporate, is the rationally reflective enterprise introduced to us by
Socrates. It is what results when the rational, normative connections
among claims that govern the practice of giving and asking for reasons
are themselves brought into the game, as liable to demands for reasons
and justification. Saying or thinking something, making it explicit,
consists in applying concepts, thereby taking up a stance in the space
of reasons, making a move in the game of giving and asking for reasons.
The structure of that space, of that game, though, is not given in
advance of our finding out how things are with what we are talking
about. For what is \emph{really} a reason for what depends on how things
\emph{actually} are. But that inferential structure itself can be the
subject of claims and thoughts. It can itself be made explicit in the
form of claims about what follows from what, what claims are evidence
for or against what other claims, what else one would be committing
oneself to by making a certain judgment or performing a certain action.
So long as the commitment to the propriety of the inference from German
nationality to barbarity and unusual cruelty remains merely implicit in
the use of term such as `Boche', it is hidden from rational scrutiny.
When it is made explicit in the form of the conditional claim ``Anyone
who is German is barbarous and unusually prone to cruelty,'' it is
subject to rational challenge and assessment; it can, for instance, be
confronted with such counterexamples as Bach and Goethe.

Discursive explicitness, the application of concepts, is Kantian
apperception or consciousness. Bringing into discursive explicitness the
inferentially articulated conceptual norms in virtue of which we can be
conscious or discursively aware of anything at all is the task of
reflection, or self-consciousness. This is the expressive task
distinctive of philosophy. Of course, the practitioners of special
disciplines, such as membrane physiology, are concerned to unpack and
criticize the inferential commitments implicit in using concepts such as
\underline{lipid soluble} with a given set of circumstances and
consequences of application, too. It is the emphasis on the ``anything
at all'' distinguishes philosophical reflection from the more focused
reflection that goes on within such special disciplines. Earlier I
pinned on Kant a view that identifies us as distinctively
\emph{rational} creatures, where that is understood as a matter of our
being subject to a certain kind of \emph{normative} assessment: we are
creatures who can undertake \emph{commitments} and
\emph{responsibilities} that are \emph{conceptually} articulated in that
their contents are articulated by what would count as \emph{reasons} for
them (as well as what other commitments and responsibilities they
provide reasons for). One of philosophy's defining obligations is to
supply and deploy an expressive toolbox, filled with concepts that help
us make explicit various aspects of \emph{rationality} and
\emph{normativity} in general. \textbf{The topic of philosophy is
normativity in all its guises, and inference in all its forms.} And its
task is an \emph{expressive, explicative} one. So it is the job of
practitioners of the various philosophical subfields to design and
produce specialized expressive tools, and to hone and shape them with
use. At the most general level, \emph{inferential} connections are made
explicit by \emph{conditionals}, and their \emph{normative} force is
made explicit by \emph{deontic} vocabulary. Different branches of
philosophy can be distinguished by the different sorts of inference and
normativity they address and explicate, the various special senses of
``if\ldots then\_\_\_,'' or of `ought' for which they care. Thus
philosophers of science, for instance, develop and deploy conditionals
codifying causal, functional, teleological, and other explanatory
inferential relations, value theorists sharpen our appreciation of the
significance of the differences in the endorsements expressed by
prudential, legal, ethical, and aesthetic `ought's, and so on.

6. I said at the outset that I thought of philosophy as defined by its
history, rather than by its nature, but that, following Hegel, I think
of our task as understanding it by finding or making a nature in or from
its history. The gesture I have made in that direction today, though,
could be also be summarized in a different kind of definition, namely in
the ostensive definition: Philosophy is the kind of thing that Kant and
Hegel did (one might immediately want to add Plato, Aristotle, Frege and
Wittgenstein to the list, and then we are embarked on the enterprise of
turning a gesture into a story, indeed, a history). So one might ask:
Why not just say that, and be done with it? While, as I've indicated, I
think that specification is a fine place to start, I also think there is
a point to trying to be somewhat more explicit about just what sort of
thing it is that one takes it Kant and Hegel (and Frege and
Wittgenstein) did. Doing that is not being satisfied just with a wave at
philosophy as something that has a history. It is trying rationally to
reconstruct that tradition, to recast it into a form in which a
constellation of ideas can be seen to be emerging, being expressed,
refined, and developed.

With those giants, I see philosophy as a discipline whose distinctive
concern is with a certain kind of \emph{self-consciousness}: awareness
of ourselves as specifically \emph{discursive} (that is,
concept-mongering) creatures. It's task is understanding the conditions,
nature, and consequences of conceptual norms and the
activities---starting with the social practices of giving and asking for
reasons---that they make possible and that make them possible. As
concept users, we are beings who can make explicit how things are and
what we are doing---even if always only in relief against a background
of implicit circumstances, conditions, skills, and practices. Among the
things on which we can bring our explicitating capacities to bear are
those very concept-using capacities that make it possible to make
anything at all explicit. Doing that, I am saying, is philosophizing.

It is easy to be misled by the homey familiarity of these sentiments,
and correspondingly important to distinguish this characterization from
some neighbors with which it is liable to be confused. There is a clear
affinity between this view and Kant's coronation of philosophy as "queen
of the sciences." For on this account philosophy does extend its view to
encompass all activity that is discursive in a broad sense---that is,
all activity that presupposes a capacity for judgment and agency,
sapience in general. But in this sense, philosophy is at most \emph{a}
queen of the sciences, not \emph{the} queen. For the magisterial sweep
of its purview does not serve to distinguish it from, say, psychology,
sociology, history, literary or cultural criticism, or even journalism.
What distinguishes it is the \emph{expressive} nature of its concern
with discursiveness in general, rather than its inclusive scope. My
sketch was aimed at introducing a specific difference pertaining to
philosophy, not a unique privilege with respect to such other
disciplines.

Again, as I have characterized it, philosophy does not play a
\emph{foundational} role with respect to other disciplines. Its claims
do not stand prior to those of the special sciences in some order of
ultimate justification. Nor does philosophy sit at the other end of the
process as final judge over the propriety of judgments and actions---as
though the warrant of ordinary theoretical and practical applications of
concepts remained somehow provisional until certified by philosophical
investigation. And philosophy as I have described it likewise asserts no
methodological privilege or insight that potentially collides with the
actual procedures of other disciplines.

Indeed, philosophy's own proper concerns with the nature of normativity
in general, and with its conceptual species in particular, so on
inference and justification in general, impinge on the other disciplines
in a role that equally well deserves the characterization of
"handmaiden." For what we do that has been misunderstood as having
foundational or methodological significance is provide and apply tools
for unpacking the substantive commitments that are implicit in the
concepts deployed throughout the culture, including the specialized
disciplines of the high culture. Making those norms and inferences
explicit in the form of claims exposes them for the first time to
reasoned assessment, challenge, and defense, and so to the sort of
rational emendation that is the primary process of conceptual evolution.
But once the implicit presuppositions and consequences have been brought
out into the daylight of explicitness, the process of assessment,
emendation, and so evolution is the business of those whose concepts
they are---and not something philosophers have any particular authority
over or expertise regarding. Put another way, it is the business of
philosophers to figure out ways to increase semantic and discursive
self-consciousness. What one does with that self-consciousness is not
our business \emph{qua} philosophers---though of course, \emph{qua}
intellectuals generally, it may well be.

Philosophy's \emph{expressive} enterprise is grounded in its focus on us
as a certain kind of thing, an expressing thing: as at once creatures
and creators of conceptual norms, producers and consumers of reasons,
beings distinguished by being subject to the peculiar normative force of
the better reason. Its concern with us as specifically \emph{normative}
creatures sets philosophy off from the empirical disciplines, both the
natural and the social sciences. It is this normative character that
binds together the currents of thought epitomized in Stanley Cavell's
characteristically trenchant aphorism that Kant depsychologized
epistemology, Frege depsychologized logic, and Wittgenstein
depsychologized psychology. We might add that Hegel depsychologized
history. The depsychologizing move in question is equally a
desociologizing. For it is a refocusing on the \emph{normative
bindingness} of the concepts deployed in ground-level empirical
knowledge, reasoning, and thought in general. This is a move beyond the
narrowly \emph{natural} (in the sense of the describable order of
causes), toward what Hegel called the `spiritual' {[}geistig{]}, that
is, the \emph{normative} order. That its concern is specifically with
our \emph{conceptual} normativity sets philosophy off from the other
humanistic disciplines, from the literary as well as the plastic arts.
Conceptual commitments are distinguished by their inferential
articulation, by the way they can serve as reasons for one another, and
by the way they stand in need of reasons, their entitlement always
potentially being at issue. Now in asserting the centrality and
indispensability, indeed, the criterial role, of practices of giving and
asking for reasons, I am far from saying that reasoning---or even
thinking---is all anyone ought to do. I am saying that philosophers'
distinctive concern is with what else those reason-mongering practices
make possible, and how they do, on the one hand, and with what it is
that makes them possible---what sort of doings count as sayings, how
believing or saying that is founded on knowing how---on the other. It is
this distinctive constellation of concerns that makes philosophy the
party of reasons, and philosophers the friends of the norms, the ones
who bring out into the light of discursive explicitness our capacity to
make things discursively explicit.
